\documentclass[11pt]{article}

\usepackage[a4paper]{geometry}
\geometry{left=2.0cm,right=2.0cm,top=2.5cm,bottom=2.5cm}

\usepackage{ctex}
\usepackage{amsmath,amsfonts,graphicx,amssymb,bm,amsthm}
\usepackage{algorithm,algorithmicx}
\usepackage{umoline}
\usepackage[noend]{algpseudocode}
\usepackage{fancyhdr}


\newcounter{counter_exm}\setcounter{counter_exm}{1}
%\newcounter{counter_thm}\setcounter{counter_thm}{1}
%\newcounter{counter_lma}\setcounter{counter_lma}{1}
%\newcounter{counter_dft}\setcounter{counter_dft}{1}
%\newcounter{counter_clm}\setcounter{counter_clm}{1}
%\newcounter{counter_cly}\setcounter{counter_cly}{1}

\newtheorem{theorem}{{\hskip 1.7em \bf 定理}}
\newtheorem{lemma}[theorem]{\hskip 1.7em 引理}
\newtheorem{proposition}[theorem]{Proposition}
\newtheorem{claim}[theorem]{\hskip 1.7em 命题}
\newtheorem{corollary}[theorem]{\hskip 1.7em 推论}
\newtheorem{definition}[theorem]{\hskip 1.7em 定义}

\renewcommand{\emph}[1]{\begin{kaishu}#1\end{kaishu}}

\newenvironment{solution}{{\noindent\hskip 2em \bf 解 \quad}}


\renewenvironment{proof}{{\noindent\hskip 2em \bf 证明 \quad}}{\hfill$\qed$\par}
\newenvironment{example}{{\noindent\hskip 2em \bf 例 \arabic{counter_exm}\quad}}{\addtocounter{counter_exm}{1}\par}

\newenvironment{concept}[1]{{\bf #1\quad} \begin{kaishu}} {\end{kaishu}\par}

\newcommand\E{\mathbb{E}}
\newcommand\Z{\mathbb{Z}}
\newcommand\N{\mathbb{N}}

\begin{document}
    
    \pagestyle{fancy}
    \lhead{\kaishu 中国科学院大学}
    \chead{}
    \rhead{\kaishu 2024年秋季学期组合数学}
    
    \begin{center}
        {\LARGE \bf 组合数学第二讲}\\
    \end{center}
    \begin{kaishu}
        授课时间: 2024年9月2日\quad
        授课教师: 孙晓明
        \hfill 记录人: 袁晨圃
    \end{kaishu}

    \section{广义二项式定理的一些应用}

    $(1+x)^n=\sum\limits_{k\geq 0}\dbinom{n}{k}x^k$

    考虑$\sum\limits_{k\geq 0}\binom{n}{3k}:$

    代入$x=1$
    \begin{align*}
        (1+1)^n=\sum_{k\geq 0}\binom{n}{k}
    \end{align*}

    代入$x=e^{i\frac{2}{3}\pi}$

    \begin{align*}
        (1+e^{i\frac{2}{3}\pi})^n&=\sum\limits_{k\geq 0}\binom{n}{k}e^{i\frac{2k}{3}\pi}\\
        &=\sum_{k\geq 0}\binom{n}{3k}+e^{i\frac{2}{3}\pi}\sum_{k\geq 0}\binom{n}{3k+1}+e^{i\frac{4}{3}\pi}\sum_{k\geq 0}\binom{n}{3k+2}\\
    \end{align*}

    代入$x=e^{i\frac{4}{3}\pi}$
    
    \begin{align*}
        (1+e^{i\frac{4}{3}\pi})^n&=\sum\limits_{k\geq 0}\binom{n}{k}e^{i\frac{4k}{3}\pi}\\
        &=\sum_{k\geq 0}\binom{n}{3k}+e^{i\frac{4}{3}\pi}\sum_{k\geq 0}\binom{n}{3k+1}+e^{i\frac{2}{3}\pi}\sum_{k\geq 0}\binom{n}{3k+2}
    \end{align*}

    注意到$e^{i\frac{4}{3}\pi}+e^{i\frac{2}{3}\pi}=-1$

    所以

    \begin{align*}
        \sum_{k\geq 0}\binom{n}{3k}&=\frac 13[2^n+(1+e^{i\frac23\pi})^n+(1+e^{i\frac43\pi})^n]\\
        &=\frac13[2^n+e^{i\frac n6\pi}+e^{-i\frac n3\pi}]
    \end{align*}

    \begin{equation}
        \label{eqsumCnkxk}
        (1+x)^n=\sum\limits_{k\geq 0}\binom{n}{k}x^k
    \end{equation}

    \textbf{考虑对\eqref{eqsumCnkxk}求导:}

    \[n(1+x)^{n-1}=\sum\limits_{k\geq 1}k\binom{n}{k}x^{k-1}\]

    可得到\begin{equation}\sum_{k\geq 0}k\binom{n}{k}=n\cdot 2^{n-1}\label{eqsumkCnk}\end{equation}

    \begin{example}
        $T\leq S=\{1,2,\cdots,n\}$,求$E(|T|)$
    \end{example}

    \begin{solution}
        \begin{align*}
            E(|T|)=\sum_{k=0}^n=k\cdot\frac{\binom{n}{k}}{2^n}=\frac n2
        \end{align*}

        法二:期望具有线性性,$E(\sum)=\sum E_i=\dfrac n2$
    \end{solution}

    \begin{example}
        求$\sum k^2\dbinom{n}{k}$
    \end{example}

    \begin{solution}
        在\eqref{eqsumkCnk}左右两边同时乘$x$再求导
    \end{solution}

    \begin{equation}
        \label{eqsumCmkxk}
        (1+x)^m=\sum\limits_{k\geq 0}\binom{m}{k}x^k
    \end{equation}

    \textbf{将\eqref{eqsumCnkxk}和\eqref{eqsumCmkxk}乘起来:}

    \[(1+x)^{m+n}=\sum_{k\geq 0}x^k\cdot\sum_{j=0}^k\binom{n}{j}\binom{m}{k-j}\]

    又\[\sum_{k\geq 0}\binom{m+n}{k}x^k=(1+x)^{m+n}\]

    所以\[\binom{m+n}{k}=\sum_{j=0}^{k}\binom{n}{j}\binom{m}{k-j}\]

    \begin{theorem}
        [范德蒙恒等式]
        \[\dbinom{m+n}{k}=\sum\limits_{j=0}^{k}\dbinom{n}{j}\dbinom{m}{k-j},\quad m,n\in\mathbb{R},k\in\mathbb{N}\]
        \label{vandermone}
    \end{theorem}
    特别地:

    令$n=m$
    \[\dbinom{2n}{k}=\sum_{j=0}^n\dbinom{n}{j}\binom{n}{n-j}=\sum_{j=0}^n\dbinom{n}{j}^2=\sum_{j\geq0}\dbinom{n}{j}^2\]

    令$m=1$
    \[\binom{n+1}{k}=\sum_{j=k-1}^k\binom{n}{j}\binom{1}{k-j}=\binom{n}{k-1}+\binom{n}{k}\]
    
    \begin{theorem}[朱世杰恒等式]
        \[\binom{k}{k}+\binom{k+1}{k}+\binom{k+2}{k}+\cdots+\binom{n}{k}=\binom{n+1}{k+1}\]
        使用下降幂的形式:
        \[\sum x^{\underline{k}}=\dfrac{1}{k+1}x^{\underline{k+1}}\]
    \end{theorem}

    \begin{proof}
        因为$\dbinom{k}{k}=\dbinom{k+1}{k+1}$,故
        \begin{align*}
            &\binom{k}{k}+\binom{k+1}{k}+\binom{k+2}{k}+\cdots+\binom{n}{k}\\
            &=\binom{k+1}{k+1}+\binom{k+1}{k}+\binom{k+2}{k}+\cdots+\binom{n}{k}\\
            &=\binom{k+2}{k+1}+\binom{k+2}{k}+\cdots+\binom{n}{k}\\
            &\vdots\\
            &=\binom{n+1}{k+1}
        \end{align*}
    \end{proof}

    特别地,令$k=2:$
    \[\binom{2}{2}+\binom 32+\binom 42+\cdots+\binom n2=\binom{n+1}{3}\]

    所以\[\sum_{k=1}^n k^2=\sum(k(k-1)+k)=\sum_{k=1}^n(2\cdot\binom{k}{2}+1\cdot\binom{k}{1})=2\cdot\binom{n+1}{3}+\binom{n+1}{2}\]

    类似地, 求\(\sum\limits_{k=1}^n k^3:\)

    待定系数,设\[k^3=c_3k^{\underline{3}}+c_2k^{\underline{2}}+c_1k^{\underline{1}}+c_0k^{\underline{0}}\]

    对于一般的$x^d:$
    \[x^d=\sum_{k=0}^d c_k x^{\underline{k}}\]

    a.k.a. 第二类斯特林数(普通幂转下降幂)

    \section{特殊排列}

    \textbf{圆排列}

    $n$个人选$k$个人圆排列:
    \[\dbinom{n}{k}(k-1)!=\dfrac{n^{\underline{k}}}{k}\]

    \textbf{可重排列}

    $k_1$个$a_1$,$k_2$个$a_2,\cdots,k_t$个$a_t$全排列,方案数:
    \[\dfrac{(k_1+k_2+\cdots+k_t)!}{k_1!k_2!\cdots k_t!}=\dfrac{(\sum k_i)!}{\prod k_i!}\]

    使用可重排列可重新解释二项式定理公式$(a+b)^n=\sum\limits_{k=0}^n\binom{n}{k}a^kb^{n-k}$
    
    (选$k$个$a$,$n-k$个$b$可重排列)

    故
    \[(a+b+c)^n=\sum\frac{n!}{i!j!(n-i-j)!}a^ib^jc^{n-i-j}\]

    一般情况:

    \[(\sum_{i=1}^n x_i)^2=\sum_{\substack{0\leq i_1,\cdots i_k\leq n\\i_1+\cdots+i_k=n}}\dfrac{n!}{i_1!i_2!\cdots i_k!}x_1^{i_1}x_2^{i_2}\cdots x_k^{i_k}\]

    引进广义组合数记号:

    \[\begin{pmatrix}n\\k,n-k\end{pmatrix}:=\dbinom{n}{k}\]

    \[\begin{pmatrix} n\\i_1,i_2,\cdots,i_n \end{pmatrix}:=\dfrac{n!}{i_1!i_2!\cdots i_k!}\]

    \section{斯特林公式}

    对$n!$的渐进估计:$n!\sim\sqrt {2\pi n}\left(\dfrac{n}{e}\right)^n$

    \begin{example}
        掷硬币问题:1000次,500次朝上概率:
    \end{example} 
    
    \[\dbinom{2n}{n}\cdot\dfrac 1{2^{2n}}\sim \dfrac{1}{\sqrt n}\]

    \begin{align*}
        &\dfrac{(2n)!}{n!n!}\cdot\dfrac1{2^{2n}}\\
        \sim\ &\dfrac{\sqrt{2\pi\cdot 2n}\left(\dfrac{2n}{e}\right)^{2n}}{[\sqrt{2\pi\cdot 2n}\left(\dfrac ne\right)]^2}\bigg/ 2^{2n}\\
        \sim\ &\dfrac{1}{\sqrt n}
    \end{align*}

    \section{整值多项式}

    \begin{definition}
        整值多项式$P(x)$:$\forall x\in\mathbb{Z}\Rightarrow P(x)\in\mathbb{Z}$
    \end{definition}

    显然,整系数多项式一定是整值多项式。

    反之不然,e.g. $\dfrac{n(n-1)}{2}$ 

    \begin{theorem}
        次数为$d$的多项式$P(x)$是整值多项式$\Leftrightarrow P(x)=\sum\limits_{0\leq k\leq d}C_k\dbinom{x}{k},C_k\in\mathbb{Z}$
    \end{theorem}

    \begin{proof}
        必要性:二项式系数是整值多项式

        充分性:

        \begin{align*}
            &P(0)\in \mathbb{Z}\Rightarrow C_0\in \mathbb{Z}\\
            &P(1)\in \mathbb{Z}\Rightarrow P(1)=C_1\binom{1}{1}+\underbrace{C_0\binom{1}{0}}_{\in\Z}\in \mathbb{Z}\Rightarrow C_1\in \Z\\
            &\vdots\\
            &P(k)\in \mathbb{Z}\Rightarrow P(k)=C_k\binom{k}{k}+\underbrace{C_0\binom{1}{0}+C_1\binom{k}{1}+\cdots}_{\in\Z}\in\Z\Rightarrow C_k\in\Z
        \end{align*}
    \end{proof}

    \begin{corollary}
        在$P(0),P(1),\cdots P(d)$上(连续$d+1$个整数点)都是整值$P(x)$的多项式是整值多项式
        \label{cor1}
    \end{corollary}

    \begin{corollary}
        在$P(m),P(m+1),\cdots P(m+d)$上(连续$d+1$个整数点)都是整值$P(x)$的多项式是整值多项式
    \end{corollary}

    \begin{proof}
        令$H(x)=P(x+m)$

        $H(0)=P(m)\in \Z, \quad H(1)=P(m+1),\cdots,H(d)=P(m+d)\in\Z$

        平移不影响次数,$\deg(H)=\deg(P)=d$,由\ref{cor1}知$H(x)$是整值多项式,设$H(x)=\sum\limits_{k=0}^d e_k\dbinom{x}{k}$则$e_i$是整数

        $H(x)=P(x+m),\quad P(y)=H(y-m)=\sum\limits_{k=0}^d e_k\dbinom{y-m}{k}$

        由 \ref{vandermone}, $\dbinom{y-m}{k}=\sum\limits_{j=0}^k\dbinom{y}{j}\dbinom{-m}{k-j}=\sum\limits_{j=0}^d\underbrace{\left(\sum\limits_{k=j}^d e_k\dbinom{-m}{k-j}\right)}_{\in \Z}\underbrace{\dbinom{y}{j}}_{\in\Z}$
    \end{proof}

    \begin{example}
        思考题:在$d+1$个整点上都是整值的多项式$P(x)$是不是整值多项式?
    \end{example}

\end{document}





